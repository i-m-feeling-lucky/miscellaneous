% !TEX program = xelatex
\documentclass[aspectratio=169,10pt]{beamer}
\usepackage[slantfont, boldfont]{xeCJK}
\setCJKmainfont{Source Han Sans SC}
\usetheme[subsectionpage=progressbar]{metropolis}
\setcounter{tocdepth}{4}
\usepackage{appendixnumberbeamer}
\usepackage{multicol}

\title{\huge 在线面试平台}
\subtitle{\small 组名:i-m-feeling-lucky}
\author{黄旭\, 秦亚璇\, 沈欣怡\, 汪若辰\, 吴舜钰\\
吴扬帆\, 徐文乐\, 余磊\, 张晏铭\, 张永停}
\date{2020.9.6}

\begin{document}

\maketitle

\setbeamerfont{section in toc}{size=\large}
\setbeamerfont{subsection in toc}{size=\footnotesize}
\setbeamertemplate{section in toc}[sections numbered]

\begin{frame}{Table of Contents}
  \tableofcontents
\end{frame}

\section{简介}
\begin{frame}{简介}
  简介
\end{frame}

\section{分工}
\begin{frame}{分工}
  分工
\end{frame}

\section{设计思路}
\subsubsection{前端}
\begin{frame}{前端}
  前端
\end{frame}
\subsubsection{后端}
\begin{frame}{后端}
  后端
\end{frame}

\section{完成情况}
\begin{frame}{完成情况}
  完成情况
\end{frame}

\section{特色}
\begin{frame}{特色}
  特色
\end{frame}

\begin{frame}[standout]
  \huge Thanks!
\end{frame}

\appendix

\begin{frame}{演示}
  \begin{center}\large \url{https://interview.yusanshi.com/}\end{center}
\end{frame}

\begin{frame}[standout]
  \huge Q\&A
\end{frame}

\end{document}
