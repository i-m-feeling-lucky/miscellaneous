% !TEX program = xelatex
\documentclass[aspectratio=169,10pt]{beamer}
\usepackage[slantfont, boldfont]{xeCJK}
\setCJKmainfont{Source Han Sans SC}
\usetheme[subsectionpage=progressbar]{metropolis}
\setcounter{tocdepth}{4}
\usepackage{appendixnumberbeamer}

\title{\huge 在线面试平台}
\subtitle{\small 组名:i-m-feeling-lucky}
\author{黄旭\, 秦亚璇\, 沈欣怡\, 汪若辰\, 吴舜钰\\
吴扬帆\, 徐文乐\, 余磊\, 张晏铭\, 张永停}
\date{2020.9.6}

\begin{document}

\maketitle

\setbeamerfont{section in toc}{size=\large}
\setbeamerfont{subsection in toc}{size=\footnotesize}
\setbeamertemplate{section in toc}[sections numbered]

\begin{frame}{目录}
  \renewcommand{\baselinestretch}{1.75}\normalsize
  \begin{columns}[t]
    \begin{column}{.3\textwidth}
      \tableofcontents[sections={-3}]
    \end{column}
    \begin{column}{.5\textwidth}
      \tableofcontents[sections={4-}]
    \end{column}
  \end{columns}
  \renewcommand{\baselinestretch}{1.0}\normalsize
\end{frame}

\section{简介}
\begin{frame}{简介}
  简介
\end{frame}

\section{分工}
\begin{frame}{分工}
  分工
\end{frame}

\section{设计思路}
\subsection{前端}
\begin{frame}{前端}
  前端
\end{frame}
\subsection{后端}
\begin{frame}{后端}
  后端
\end{frame}

\subsection{API}
\begin{frame}{API}
  API
\end{frame}

\subsection{权限控制}
\begin{frame}{基于 token 的认证系统*}

\end{frame}

\begin{frame}{最小权限原则}

\end{frame}

\subsection{面试}
\begin{frame}{面试}
  面试
\end{frame}

\section{完成情况}

\subsection{基础功能}
\begin{frame}{基础功能}
  基础功能
\end{frame}

\subsection{额外功能}
\begin{frame}{额外功能}
  额外功能
\end{frame}

\section{其他}

\subsection{工作流}

\begin{frame}{Git 协作}
  前端 \\
  test branch \\
  后端
\end{frame}

\begin{frame}{Issue}
  Issues
\end{frame}

\begin{frame}{Pull Request}
  Pull Request
\end{frame}

\begin{frame}{进度跟踪}
  \begin{figure}
    \includegraphics[width=0.85\textwidth]{../images/all-projects-wip.png}
    \caption{使用 GitHub 的 Projects 功能跟踪每个项目的进度。}
  \end{figure}
\end{frame}

\begin{frame}{进度跟踪}
  \begin{figure}
    \includegraphics[width=0.75\textwidth]{../images/backend-project-wip.png}
    \caption{以 Backend 项目为例,这是它在某一时刻的状态。}
  \end{figure}
\end{frame}

\subsection{测试}
\begin{frame}{测试}
  测试
\end{frame}

\subsection{CI/CD}
\begin{frame}{CI/CD}
  CI/CD
\end{frame}

\subsection{文档}
{
  \usebackgroundtemplate{\includegraphics[width=\paperwidth]{../images/doc-main-page.png}}
  \begin{frame}[plain]
  \end{frame}
}

\begin{frame}[standout]
  \huge Thanks!
\end{frame}

\appendix

\begin{frame}{演示}
  \large
  主站\\
  \url{https://interview.yusanshi.com/} \\
  \bigskip
  文档\\
  \url{https://i-m-feeling-lucky.github.io/}
\end{frame}

\begin{frame}[standout]
  \huge Q\&A
\end{frame}

\end{document}
